\documentclass[a4paper, 12pt]{article}
\usepackage[utf8]{inputenc}
\usepackage{graphicx}
\usepackage{amsmath}
\usepackage[bottom=2.0cm,top=2.0cm,left=2.0cm,right=2.0cm]{geometry}
\usepackage[brazillian]{babel}
\usepackage{indentfirst}
\usepackage{hyperref}  %%%%
\usepackage{alltt}
\setlength{\parindent}{0.5cm}
\hypersetup{colorlinks,citecolor=black,filecolor=black,linkcolor=black,urlcolor=black} %%%%


\begin{document}
\title{Capa}

\begin{titlepage}
	\begin{center}
		\begin{figure}[htb!]
			\begin{flushleft}
				\includegraphics[width=3.9cm]{imagens/logo_unb.png}
			\end{flushleft}
		\end{figure}
        \vspace{-2.5cm}
        \hspace{2.1cm}\Large{\textbf{Universidade de Brasília}}\\
        
        \vspace{200pt}
        
        \LARGE{\textbf{Relatório:}}\\ 
        \Large{Formalização de Propriedades do Algoritmo Radix Sort}\\
        
        \vspace{150pt}
        
        \hfill Grupo: \\
        
        \vspace{30pt} 
        \hfill Leonardo Ribas do Nascimento\hspace{20pt} Mat: 17/0038963\\
        \hfill Amanda Augusto da Silva\hspace{20pt} Mat: 18/0012053\\

        \vspace{25pt}
        \hfill \underline{Professor:}\\
        \hfill Flávio L. C. de Moura\\ 
        
        
        \vspace{\fill}
        \LARGE \bf{\today}
          
	\end{center}
\end{titlepage}


\newpage


\pagenumbering{arabic}
\large
\section{Introdução}
O projeto desenvolvido ao longo da disciplina de Lógica Computacional 1 visa a demonstração da correção do algoritmo de ordenação Radix Sort. Tal demonstração consiste na formalização de três propriedades do algoritmo utilizando o assistente de provas PVS, assim como técnincas dedutivas da lógica de predicados.

\section{Especificação do Problema}
O Radix sort utilizado no projeto ordena números inteiros, partindo da ordenação do dígito menos significativo ao mais significativo utilizando o merge sort como algoritmo auxiliar.

\begin{figure}[!h]
    \centering
    \includegraphics[width=0.7\textwidth]{radixsort.png}
    \caption{Funcionamento do Merge Sort}
\end{figure}

A primeira questão, consiste em demonstrar que se uma lista está ordenada até o d-ésimo dígito,
e se aplica radixsort entre os dígitos d e k, então ficará ordenada até o dígito k + 1.

A segunda questão, está relacionada com o fato de que a função radixsort preserva os elementos das listas dadas como argumento

Finalmente, a terceira questão é demonstrar que função radixsort gera uma lista ordenada

\section{Descrição das soluções e formalizações}

\subsection{Questão 1}
Queremos provar a seguinte propriedade:
\begin{alltt}
 FORALL(l : list[nat], k : nat, d:nat | d <= k) :
     is_sorted_ud?(l,d) =>
        is_sorted_ud?(radixsort(l, k, d), k+1)
\end{alltt}


Iniciamos a prova com indução forte no intervalo k-d e obtemos como hipótese de indução que a propriedade vale para qualquer intervalo menor que k-d.
Ficamos então com o seguinte sequente:


\begin{alltt}
 [-1]  FORALL (y_1: nat, y_2: {d: nat | d <= y_1}):
        FORALL (l: list[nat]):
          y_1 - y_2 < x!1 - x!2 IMPLIES
           is_sorted_ud?(l, y_2) =>
            is_sorted_ud?(radixsort(l, y_1, y_2), y_1 + 1)
 [-2]  is_sorted_ud?(l, x!2)
  |-------
 [1]   is_sorted_ud?(radixsort(l, x!1, x!2), x!1 + 1)
\end{alltt}


Aplicamos a definição de radixsort na formula 1, e dividiremos nossa prova em 2 subprovas de acordo com a implementação do próprio radixsort.
A primeira subprova assume que k e d são iguais (PVS os denomina x!2 x!1 respectivamente).
Pela definição do radixsort quando k e d são iguais é retornado apenas um merge\_sort(l, d), ficamos com o seguinte sequente:


\begin{alltt}
 [-1]  x!2 = x!1
 [-2]  is_sorted_ud?(l, x!2)
  |-------
 [1]   is_sorted_ud?(merge_sort(l, x!2), 1 + x!1)
\end{alltt}


Essa subprova é fechada utilizando-se o lema "merge\_sort\_d\_sorts" que afirma que qualquer lista ordenada pelo merge-sort até o d-ésimo digito tambem está ordenada até o d-ésimo digito + 1.
Na segunda subprova temos que $k \neq d$ e pela definição do radixsort temos que provar:


\begin{alltt}
 is_sorted_ud?(radixsort(merge_sort(l, x!2), x!1, 1 + x!2), 1 + x!1)
\end{alltt}


Temos por hipótese de indução que para qualquer lista e qualquer intervalo menor que k-d a propriedade vale, portantao podemos instanciamor nossa hipótese como se segue: \\
\begin{itemize}
\item l recebe merge\_sort(l, x!2),\\
\item y\_2 recebe 1+x!2,\\
\item y\_1 recebe x!1,\\
\item pois $x!1 - (1+x!2) < x!1 - x!2$.
\end{itemize}
 
 
Ficamos com o seguinte sequente:

\begin{alltt}
 is_sorted_ud?(merge_sort(l, x!2), 1 + x!2) =>
        is_sorted_ud?(radixsort(merge_sort(l, x!2), x!1, 1 + x!2), x!1 + 1)
\end{alltt}


Portanto se provarmos o antecedente is\_sorted\_ud?(merge\_sort(l,x!2), 1 + x!2) fecharemos a prova. E provamos isso da mesma maneira que provamos a primeira subprova do problema.  

\subsection{Questão 2}
Queremos provar a seguinte propriedade:


\begin{alltt}
FORALL(l : list[nat], k : nat, d : nat | d <=k):
  permutations(l, radixsort(l,k,d))
\end{alltt}


Assim como na propriedade anterior, aplicamos indução forte no intervalo k-d, e dividiremos nossa prova em duas subprovas: caso $k = d$ e caso $k \neq d$.


\item Caso $k = d$:\\
Por definição, quando $k = d$ radixsort retorna merge\_sort(l, x!2), e portanto temos que provar:


\begin{alltt}
permutations(l, merge_sort(l, x!2))
\end{alltt}

Provamos isso a partir do lema "merge\_sort\_permutes" que afirma que o merge\_sort preserva os elementos da lista.

 Caso $k \neq d$:\\
Instanciamos nossa hipótese de indução da mesma maneira que instanciamos na questão anterior e obtemos:


\begin{alltt}
{-1}  permutations(merge_sort(l, x!2),
                   radixsort(merge_sort(l, x!2), x!1, 1 + x!2))
  |-------
[1]   x!2 = x!1
[2]   permutations(l, radixsort(merge_sort(l, x!2), x!1, 1 + x!2))
\end{alltt}


Sabemos que pelo lemma "merge\_sort\_permutes" merge\_sort(l, x!2) é uma permutação de l. 


\begin{alltt}
{-1}  permutations(l, merge_sort(l, x!2))
[-2]  permutations(merge_sort(l, x!2),
                   radixsort(merge_sort(l, x!2), x!1, 1 + x!2))
  |-------
[1]   x!2 = x!1
[2]   permutations(l, radixsort(merge_sort(l, x!2), x!1, 1 + x!2))

\end{alltt}


E pelo lema "permutations\_is\_transitive" temos que se l permuta com merge\_sort(l, x!2) e que merge\_sort(l, x!2) permuta com radixsort(merge\_sort(l, x!2), x!1, 1+x!2) então l permuta com radixsort(merge\_sort(l, x!2), x!1, 1 + x!2).


\begin{alltt}
{-1}  permutations(l, merge_sort(l, x!2))
[-2]  permutations(merge_sort(l, x!2),
                   radixsort(merge_sort(l, x!2), x!1, 1 + x!2))
  |-------
[1]   x!2 = x!1
[2]   permutations(l, radixsort(merge_sort(l, x!2), x!1, 1 + x!2))

\end{alltt}

Assim fechamos a prova da segunda propriedade.

\section{Conclusão}
Com esse projeto tivemos a oportunidade de colocar em prática os fundamentos explorados em sala de aula na disciplina de Lógica Computacional 1, completando a formalização da carreção do algoritmo radix sort.
Alem de revisitarmos conceitos de indução, utilizamos conceitos da lógica dedutiva e lógica de predicados.

As dificuladades em relação ao projeto giram em torno, principalmente, da ferramenta PVS, que a principio não dialoga diretamente com os conceitos vistos em sala. É necessário um certo período de adaptação até que se domine os conceitos básicos da ferramenta assim como a propria linguagem utilizada no PVS.

No geral a experiência em relação ao projeto foi boa, embora não tenhamos conseguido completar a questão 3, finalizar e formalizar propriedades num ambiente que garante a correção de tais provas é gratificante.

\section{Referências}
M. Ayala-Rincón and F. L. C. de Moura. Applied Logic for Computer Scientists - Computational
Deduction and Formal Proofs. Undergraduate Topics in Computer Science.
Springer, 2017


\vspace{50pt}



\end{document}
